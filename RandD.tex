\section{Current S\&C R\&D}
\noindent{(AY, Elmer)}

The main thrust of the R\&D effort of the project is to control the rate of growth of computing used.
High-Luminosity is the least pleasant way to go exploring. Unlike High(er) Energy, one has to cope with increased event size (due to pile-up), pile-up complexity increases due to many overlapping events, data set size increases due to long running period required, impacting CPU, storage and network resources. 
For example, an increase in number of PU events from 10-25 was measured to result in increase of a x3 of event reconstruction time. Such size increases (or larger) are unaffordable and must be prevented.

There are a few possible ways to deal with this set of challenges. 
\begin{itemize}
\item reduce size/event e.g. miniAOD and beyond. Study operational and Physics implications.
\item speed up reconstruction time, using vectorization and parallelization enabled by the new computing architectures. 
\item explore the limitations of AAA. Start on US-side.  
\end{itemize}

