\documentclass[11pt,a4paper]{article}

\textwidth 6.5 true in
\textheight 8.5 true in
\oddsidemargin 0 true in
\evensidemargin 0 true in
\topmargin -0.25 true in

\usepackage{authblk}
\usepackage{graphicx}
\begin{document}

\title{NSF Cooperative Agreement - Computing Section}

\author[1]{Lothar~Bauerdick}

\author[2]{Ken~Bloom}

\author[3]{Sridhara~Dasu}

\author[4]{Peter~Elmer}

\author[5]{David~Lange}

\author[6]{Kevin~Lannon}

\author[7]{Salvatore~Rappoccio}

\author[1]{Liz~Sexton-Kennedy}

\author[8]{Frank~Wuerthwein}

\author[8]{Avi~Yagil}

\affil[1]{Fermi National Accelerator Laboratory}
\affil[2]{University of Nebraska -- Lincoln}
\affil[3]{University of Wisconsin -- Madison}
\affil[4]{Princeton University}
\affil[5]{Lawrence Livermore National Laboratory}
\affil[6]{University of Notre Dame}
\affil[7]{State University of New York -- Buffalo}
\affil[8]{University of California -- San Diego}

\renewcommand\Authands{ and }

\maketitle

\newpage

\section{Software and Computing}

\subsection{Introduction}

The NSF support for software and computing at U.S. universities
played a crucial role in the success of the CMS program, having
contributed to almost all the published work thus far, including the
discovery of the Higgs boson that completed the Standard Model of
particle physics.  Continued NSF support for software and computing is
mandatory for future successes, including the possible discovery of new
physics.  In this section we briefly describe the current status and
future plans of the U.S. CMS software and computing project, focusing on
its Tier-2 program, on which U.S. and international CMS physicists rely
for extracting physics from the expected large CMS datasets.

The scale of computing resources necessary is directly coupled to the
foreseen output from the detector.  The trigger rates have been increased
by an order of magnitude compared to the original goals at the time of CMS
computing technical design report. The discovery of Higgs at low mass and
continued investigation of electroweak scale physics requires low trigger
thresholds.  During the recently started new phase of data acquisition,
Run~2 (2015-18) and Run~3 (2021-23) of LHC, about 300 fb$^{-1}$ will be
accumulated. This 300~fb$^{-1}$ dataset presents two orders of magnitude
increase in data volume compared to the Run~1 (2009-12) dataset.  An order
of magnitude arises from the increase in integrated luminosity, a factor of
three comes from the increased trigger output rate to facilitate continued
access to electroweak scale physics, and a final factor of three or so
increase is due to greater event-complexity from increased energy and
instantaneous luminosity leading to event pileup.  As the beam energy has
reached more or less the maximum expected, it is highly likely that from
here on out analyses will tend to use all the data accumulated over time,
from the 3 fb$^{-1}$ at 13 TeV accumulated in 2015 to the 300 fb$^{-1}$
expected by the end of Run~3 in 2025.  
While the analysis and Monte Carlo
production computing needs scale roughly linearly with integrated
luminosity, the reconstruction time per event for both data and simulation
grows roughly exponentially with instantaneous luminosity.  Unfortunately,
Moore's law scaling of computing capabilities and evolution of storage have
slowed down from $\times 2$ gains every 15-18 months ten years ago to a
modest $\times 2$ gain every 4-7 years expected in the future.  As a
result, the overall computing needs outpace expected technological advances
and reasonable funding scenarios significantly by the end of Run~3, with even
larger shortfalls projected for the HL-LHC era.
%requiring significant innovations in order to guarantee that the
%physics potential of the data taken is fully realized.
%Innovations in resource utilization,
%adaptation to modern computing architectures, and improved workflows,
%will need to make up for the limitations in raw scaling of resources. We
%briefly describe these evolutionary changes in the offing and project
%how agile computing, utilizing owned, opportunistic and commercial
%cloud resources, with dynamic data management and just-in-time data
%movement over wide-area networks, will work to meet our challenge. 

Our vision for meeting this challenge 
%of growth of computing needs
%beyond what is affordable via a simple Moore's law extrapolation 
is threefold. First, we will gain efficiencies by being  more
agile in the way we use the traditional Fermilab-based Tier-1 and seven
university-based Tier-2s center resources. Second, we will grow the
resource pool by more tightly integrating resources at all other
U.S.~CMS universities, DOE and NSF supercomputing centers, and
commercial cloud providers as much as possible. 
In this
context, effort funded via the Tier-2 program will become responsible to
maintain infrastructure in the Science DMZs of U.S.~CMS member institutions
jointly with IT professionals at those institutions.  The effort funded via
this proposal will provide consultation to campus IT organizations and
ultimately maintain services on hardware inside the various Science DMZs,
in order to support the desired integration of campus IT with CMS IT.
Finally, we will
pursue an aggressive R\&D program towards improvements in software
algorithms, data formats, and procedural changes for how we analyze
the data we collect and simulate, in order to contain the growth in
computing needs.

% oblivious of the computing resource provisioning.
%For the next five years, we propose to capitalize on NSF investment in
%networking at U.S. universities as well as developments from other NSF
%projects [AAA, gWMS, OSG, PRP] to integrate resources at U.S.~CMS member
%institutions beyond Tier-1 and Tier-2 centers much more tightly into the
%centrally operated services infrastructure of the CMS experiment.  



%The central services provided by those supported in this project
%will be providing seamless access to distributed world-wide 
%resources to all US CMS universities.  In this context the 
%Tier-2s will become responsible to maintain simple ``headnodes'',
%which provide seamless access or provide consultation for small 
%installations, which could serve as portals to the campus
%resources at all US CMS institutions, democratizing access to
%computing resources.  

The primary goal of our program is to empower physicists at all 48 U.S.~CMS
member institutions to conveniently analyze CMS data. 
This proposal thus focuses on the NSF supported university-based computing,
especially for the most diverse %chaotic
physicist-driven scientific data analysis activities, as they will require
the bulk of the S\&C resources requested in this proposal.  Other supported
activities are also discussed, including a brief look at the
computing R\&D necessary for the HL-LHC phase (2025+), during which another
order of magnitude in data volume (3000 fb$^{0-1}$) is expected.

\subsubsection{CMS computing model details}
\label{computingmodeldetails}
The tiered computing model of the LHC experiments, based on a distributed
infrastructure of regional centers outlined by the MONARC project {\bf
  ref}, includes a Tier-0 center at CERN, seven Tier-1 centers (including
one at Fermilab) and about fifty Tier-2 centers (including eight in the
U.S., seven of which will be supported by this award).~\footnote{The
  Fermilab facilities are covered by S\&C WBS 1; they receive no NSF
  support and are not discussed further in the proposal.  The eighth
  U.S. Tier-2 center, at Vanderbilt, is supported by the DOE Nuclear
  Physics program.}  The original model envisioned that only a very
specific set of tasks would be performed at each tier of the
infrastructure, but in recent years, much more flexibility has been
introduced into the system.  The technical changes leading to this have
largely been pioneered within the U.S., in many cases by university-based
personnel supported by the previous NSF award for U.S. CMS operations,
especially those at the Tier-2 centers.  Some of the innovations have
included commissioning the global data transfers matrix across all Tier-1
and Tier-2 sites, initiating the centralized Monte Carlo production,
leading the commissioning of the grid worldwide for CMS, introducing the
concepts of late-binding WMS, CMS data federation, dynamic data placement,
MiniAOD, and global queue across analysis and production processing.

CMS uses a variety of data formats tailored for different purposes.
Ongoing Run~1 analyses are based on the Run 1 AOD format, comprising 220~kB
per event on average.  Average AOD event sizes in Run~2 are roughly
500~kB. The growth is partly due to increased pileup, and partly by design,
to allow re-running particle-flow reconstruction in its entirety from the
AOD.  In addition, the high level trigger output rate was increased by a
factor of three between Run~1 and Run~2.  To contain costs and speed up
physics analysis, U.S.~CMS initiated the introduction of a refined analysis
format called ``MiniAOD''.  Implemented in collaboration with CERN, this
format is $1/10$ the size of the AOD, and typically requires somewhere
between $1/2$ to $1/5$ the CPU power to analyze. 
%These reductions are
%accomplished by storing only more refined information, which thus requires remaking
%the MiniAOD in response to improved calibrations, or significant
%improvements of physics objects. During Fall 2015 CMS went through two
%iterations of MiniAOD, with at least one more to follow before Moriond 2016
%conference results.  
MiniAOD is expected to satisfy the needs of at least
90\% of all physics analyses, {\it i.e.} it is envisioned that a few analyses
will require more detailed information not present in the MiniAOD, and will
thus need to access the AOD or maybe even the RAW format.
The transformation from ``MiniAOD'' to custom data for further user
analysis is necessary but has the drawback that non-negligible amounts 
of disk space at Tier-2s also needs to be provisioned to host that custom data. 

%The operational model today is that analysis groups process these primary
%data samples to produce custom Ntuples for their analyses at a event processing
%rate of 1-10~Hz or so. The custom Ntuples are then typically analyzed at
%$\times 100$ larger event rates. The transformation from primary data useful to the
%entire collaboration to custom data useful only to a handful analyses is
%thus dramatically accelerating science and reducing computing costs with
%the drawback that non-negligible amounts of disk space at Tier-2s need to
%be provisioned to host that custom data. 
%This belongs somewhere later, not sure where yet.
% US-CMS organizes this by assigning a fixed set of University groups to
% each Tier-2 as their host Tier-2 for these custom data samples.

\subsection{University Facilities (WBS 2)}

\subsubsection{Current Status}

The primary university-based facilities are the Tier-2 centers at Caltech,
Florida, MIT, Nebraska, Purdue, UC San Diego and Wisconsin.  Resources
available at these centers funded through prior NSF support are summarized
in Table~\ref{current-resources}.  Faculty collaborations at the university
level have brought access to significant opportunistic resources on their
campuses and on the cloud ($\sim$37\% of the total available to CMS).  Often
cost of infrastructure at the university Tier-2s is subsidized and the cost
of personnel is lower than at DOE labs.  Further, friendly competition amongst the
sites results in increased productivity of U.S. Tier-2s overall.  
Finally, the US CMS Tier-2 institutions also provide connections to strong
physics groups at those universities as well, resulting in prompt feedback
for operations with their intense analysis activities.

The seven
U.S. Tier-2s rank amongst the top ten providers of the 50 such CMS centers
world-wide. Together they provide about 35\% of CMS Tier-2 resources.  The
compute resources at Tier-2s serve both production and physicist analysis
cases.  The resource utilization at the US Tier-2s in the past month adds
up to about 30,000 jobs in steady state split equally between production
and analysis workflows.  These centers together are hosting 10 PB of 
centrally and 4 PB of user produced data on their storage systems.

\begin{table}
\begin{center}
\begin{tabular}{|l|c|l|c|}
\hline
\multicolumn{2}{|c|}{Number of Job Slots} &  \multicolumn{2}{|c|}{Storage (PB)} \\ \hline
Available as of 11/2015 &          &               &    \\
Purchased     & 40,698             & Purchased     & 14 \\
Opportunistic & 24,502             & Opportunistic & -  \\
Total         & 65,200             & Total         & 14 \\ \hline
Average Usage (11/2015) &          &               &    \\
Production    & 15,000             & Hosted        & 10 \\
Analysis      & 15,000             & Local         &  4 \\
Total         & 30,000             & Total         & 14 \\ \hline
\end{tabular}
\caption[]
{
Currently available and used number of job slots and storage totals
for all the US Tier-2 sites.
}
\label{current-resources}
\end{center}
\end{table}

Not only does the U.S. CMS Tier-2 program provide the premier Tier-2
centers in all of CMS, the staff at these centers, working in partnerships
with their universities, also provide the intellectual leadership to make
radical changes such as those mentioned in 
Section~\ref{computingmodeldetails}.  With the current proposal we continue
this transition, focusing on maintaining leadership in operations of a
production infrastructure, and leadership in engineering changes to afford
doing ever more demanding physics with fixed hardware budgets.  Thus, the
investment in personnel at the universities sites brings at least as much
value as the investment in the computing equipment.

\subsubsection{Future Plans}

\noindent{\bf Tier-2 Personnel:}
We continue to request in this proposal 2 FTE at each of the seven Tier-2
institutions.  On average, 1.6 FTE of these are necessary at each center to
provide high-quality service that results in very high availability,
upwards of 95\%. The personnel are responsible for all aspects of
provisioning these resources, from specifications through deployment to
operations, taking advantage of local considerations.  The remaining
$7 \times 0.4$ FTE of effort take on other roles within the larger U.S.~CMS
software and computing project as described in the later sections of
the proposal.  CMS benefits from this close connection because of
innovations and pioneering deployments initiated at U.S. Tier-2s, such as the
most recent work in testing and commissioning of the world-wide CMS data
federation using AAA technologies.

\noindent{\bf Tier-2 Resources:}
We use a simple model that scales the present usage of job slot count 
and storage to future years based the expected LHC  luminosity plan 
(see Table~\ref{projection}).
The CPU requirements are estimated in units of number of 
batch slots needed and the storage is determined by the amount of
data anticipated to be accumulated and equal amount of MC simulated.
In addition to the MiniAOD, we expect to require disk space to accommodate 
$\sim$10\% of the data in AOD format on disk to allow the $\le$ 10\% of 
analyses that require such detail.

In Table~\ref{projection}, we start by showing the actual numbers for
2015 scale of operations.  We have accumulated about 3~fb$^{-1}$
of data from LHC runs and produced 10~fb$^{-1}$ worth simulated data.
Including the data simulated and accumulated in Run~1, we are running
30,000 production jobs on 1~pb of data hosted at U.S. Tier-2s currently,
equally split between production and analysis jobs.  We show the
current average availability of job slots and storage at the U.S. Tier-2s 
to start the scaling.  We then extrapolate to out years using the
LHC run plan, which sets the scale for integrated luminosity.

We assume that the production job slot usage scales as a function of 
incremental luminosity expected that year.  The analysis job slot usage
scales as a function of the accumulated by that time, because the analysts
will be combining all of the 13~TeV data in Run~2 and Run~3.
The MiniAOD format data is expected to be
used by most analysts reducing the needed storage volume.  
We multiply this expected MiniAOD volume by 1.5 to account for the
need to support CMS upgrade activities, custom physics Ntuples, staging space for processing and reprocessing, and to allow for some
replication of popular datasets to improve access and redundancy.
To arrive at the total disk space across the 7 Tier-2's in the US, we further assume that 10\% of the AOD is also on disk.

In this simplistic projection, we have not accounted for increased reconstruction times per event
as a result of increased complexity of the events due to increased pile-up
at higher instantaneous luminosity, nor the corresponding increase in event sizes.
Nor have we made any assumptions about software performance improvements, or other increased efficiencies,
are taken into account decommissioning of old hardware. The model is thus admittedly subject to some uncertainties,
though probably no more so than the expected luminosity profile of the LHC, which we take as an input.

To meet the needs as calculated with this model requires an average annual hardware 
investment per Tier-2 between \$300k and \$500k depending on whether we assume a 
20\% or 10\% cost reduction per year due to Moore's law. The hardware budget in the present
proposal matches these needs for the lower end of this range. 

\begin{table}
\begin{center}
\begin{tabular}{|l|c|c|c|c|c|c|c|c|}
\hline
&\multicolumn{2}{|c|}{ } &\multicolumn{4}{|c|}{\bf Total for US Tier-2s}&\multicolumn{2}{|c|}{ } \\ \cline{4-7}
&\multicolumn{2}{|c|}{\bf Luminosity (fb$^{-1}$)}&\multicolumn{2}{|c|}{\bf Job Slots}&\multicolumn{2}{|c|}{\bf Storage (pb)}&\multicolumn{2}{|c|}{\bf Per Tier-2} \\ \cline{2-9}
{\bf Year}&{\bf Incr.}&{\bf Cumul.}&{\bf Prod.}&{\bf Ana.}&{\bf AOD}&{\bf MiniAOD}&{\bf Job Slots}&{\bf Storage (pb)} \\  \hline
2015&    3&   3& 15000&   15000& 3.8&  13&   5814&   2.0 \\ \hline
2016&   37&  40& 18000&   20000&   9&  31&   5429&   5.7 \\ \hline
2017&   40&  80& 20000&   40000&  18&  62&   8571&    11 \\ \hline
2018&   40& 120& 20000&   60000&  27&  92&  11429&    17 \\ \hline
2019&    0& 120& 10000&   60000&  27&  92&  10000&    17 \\ \hline
2020&    0& 120& 10000&   60000&  27&  92&  10000&    17 \\ \hline
2021&   60& 180& 30000&   90000&  40& 138&  17143&    25 \\ \hline
2022&   60& 240& 30000&  120000&  54& 184&  21429&    34 \\ \hline
2023&   60& 300& 30000&  150000&  67& 230&  25714&    42 \\ \hline
\end{tabular}
\caption[]
{
Projection of resources by the year from actual usage in 2015 for
out years through 2023, based on LHC luminosity expectation is
shown. 
%The cost of resources which satisfy the projected needs, 
%including current Moore's law scaling, i.e., 20\% to 10\% reduction 
%in costs per year, ranges between \$300K and \$500K.
}
\label{projection}
\end{center}
\end{table}

%In order to estimate the cost of provisioning the estimated needs in
%Table~\ref{projection}, we start with the recent actual incurred costs 
%for hardware purchases, which is \$118 per job slot and \$56 per TB.
%It is well known that Moore's law scaling has slowed down.  Based
%on recent trends we anticipate 20\% to 10\% reduction in cost of
%provisioning resources in the out years.  With these assumptions
%we estimate that cost of provisioning resources at U.S. Tier-2s
%is an average of \$308K and \$473K per year. 

The network bandwidth requirement will also scale with increased data size
and wide-area distributed computing.  Typically sites are connected through
100~Gbps network presently, and we expect multi-100~Gbps connections in the
coming years. Up to now, networking at Tier-2 centers has always been
funded via sources outside the U.S. CMS software and computing project. We
expect this to stay that way, and are thus not budgeting any costs for
networking as part of this proposal.

In the remainder of the proposal, we will, among other items, describe R\&D investments 
towards cost savings that are essential for the HL-LHC. If we successfully
transition some of this R\&D into production already for Run 3, then this will allow us to tolerate
the slower Moore's law curve of only 10\% per year cost reductions that is presently
deemed to be the more likely in projections at CERN.

%This bottoms-up estimate of resource needs and costs are significantly
%higher than desired.  The proposed activities in the following sections
%will develop several ideas to mitigate the growing needs by innovation.
%Use of non-traditional resources is the primary future goal.  We note that
%in recent years significant resources were available opportunistically
%through the Tier-2s.  We anticipate to increase this pool in the future.
%However, we note that innovation needed to use a variety of resources and
%environments requires R\&D that is described in later sections of the proposal.

%Two staff members are required to operate each site, to provide full
%coverage.  However, each center can be operated with approximately 1.6~FTE
%of effort.  Each site thus contributes effort to other S\&C activities
%described in the proposal.  Many site staff members are former HEP
%physicists, who have now become experts in computing. They are able to
%provide wide-ranging expertise in physics software development.

\noindent{\bf Non-traditional resources:}
In the past, resources beyond the traditional Tier-1 and Tier-2 sites were
generically lumped into the category of Tier-3.  The prevailing model of
these resources was that they were structured more or less as small Tier-2
sites operated independently of the Tier-2 program by dedicated local
administrators.  In most places, these Tier-3's were either poorly, or not at all integrated
within the larger context of campus IT infrastructures due to the idiosyncrasies and lack of
agility of the technologies available to CMS.

As these technologies have advanced, and NSF-ACI has made substantial
investment into networking infrastructure across more than 100 campuses nationwide,
we are proposing here a significantly more integrated Tier-3 program.

The Tier-3 of the future functions more as
a portal that integrates campus IT infrastructure with global CMS infrastructure seamlessly.  
It provides a local portal for university researchers to access their campus IT resources as well as 
larger-scale U.S. CMS computing resources.  It also provides a portal for the
CMS central computing infrastructure to access campus computing resources to the extend allowed by local policies.
This is accomplished, following approaches being pioneered by the 
NSF-funded ``Pacific
Research Platform'' and the CMS Connect effort which utilizes the OSG's
CI-Connect platform, to minimize administrative effort while
maximizing flexibility. 

The central hub of our proposal is a ``Tier-3 in a box''  that will be deployed at
each participating institution.  This node is a single, self-contained appliance that when deployed into a
campus Science DMZ will bridge the CMS and campus infrastructures.
This node will provide interactive data analysis, batch submission, CVMFS
software cache, XRootD data cache, and XRootD server to export local
data.  The HTCondor batch systems implemented on these nodes are all
connected to the global CMS HTCondor pool via glideinWMS.  Similarly any
University computing resources are integrated requiring nothing more than
ssh access to a U.S. CMS account on the local university cluster.  Local CMS
university groups will thus be empowered to transparently use any and all
local resources the university allows them to share in combination with the
entire Tier-1 and Tier-2 system. Official CMS data is cached locally by the
node as needed to increase processing efficiency.  Private data produced by the local university group is
served out to the Tier-1 and Tier-2 system via the XRootD server integrated
into the node.  Any data, privately or centrally produced, will thus be available anytime anywhere.

We are proposing to scale this ``Tier-3 in a box'' model to serve
as many U.S.~CMS institutions as possible but focussing initially on 25 institutions 
that have received ScienceDMZ funding from NSF-ACI
since 2012. The hardware costs as well as the human effort to deploy and
operate this system will be borne out of the University Facilities portion
of this proposal.  The entire system of Tier-3 services across all institutions
will thus be centrally managed by Tier-2 personnel such that local IT effort at the 25+ institutions
is restricted to managing local accounts, initial hardware deployment, and hardware replacement
upon failures.
At a cost of $\sim$\$10,000 per Tier-3 in a box, this is
a modest fraction of the total Tier-2 hardware budget across the seven
Tier-2s and the five years of this proposal.  We fully understand that the
above model will not be appropriate for all collaborating institutions
within U.S. CMS. We thus augment it with an additional hosted service -- CMS
Connect -- built on the OSG Connect/CI Connect model pioneered by the
University of Chicago OSG/ATLAS group.  Development work on CMS Connect is
led by Notre Dame.

Finally, in the latter years of this proposal, we will fully integrate cloud services access into this
infrastructure in such a way that local university groups can use local
funds to purchase cloud resources to augment their personal access to
computing resources, and thus accelerate their science. 
In addition to all of the above functionality geared towards data
analysis, we propose to also integrate Supercomputing resources at DOE
and NSF funded national facilities mostly for the purpose of
simulation and reconstruction, i.e. the production of the official CMS
datasets.  We expect to be collaborating on this functionality with the HEPCloud project at
Fermilab as well as the Open Science Grid (OSG).

\subsection{Operations (WBS 3)}

In addition to operating the Tier-2 facilities, personnel supported by this
project contribute to the operations of the distributed computing system of
the CMS experiment.  The tasks performed by these staff members support the
efficient processing of data and successful execution of both production
and analysis computing jobs.

\subsubsection{Current Status}

U.S. CMS personnel fill a variety of roles in CMS computing operations.
MIT staff support Tier-0 operations for the experiment, overseeing the
day-to-day operation of the facility, which is of critical importance.  Other 
MIT personnel play leading roles in
operating the experiment's data transfer system and providing support for
the distributed grid infrastructure.  UCSD maintains the CMS job submission
infrastructure.  Nebraska provides support for AAA operations and for
network performance reliability.  Johns Hopkins supports the operation of
the Frontier system that provides run conditions and other configuration
information for reconstruction and analysis jobs running on the distributed
infrastructure.  Florida takes responsibility for software distribution
throughout the grid sites of the experiment via the CVMFS caching system.

\subsubsection{Future Plans}

All of these activities are expected to continue in the coming years, as
they will always be necessary to the operation of the experiment.  They
will become even more critical to the success of CMS as the number of sites
(including opportunistic sites) grows and highly distributed storage access
over the WAN using AAA increases.  Additional operations support for smooth
operation of U.S. university portals (Tier-3-in-a-box) and efficient
harnessing of opportunistic resources is also anticipated.

\subsection{Computing Infrastructure and Services (WBS 4)}
\label{cis}

CMS as a global experiment depends on a variety of computing infrastructure
and services (CIS), several of which have been long-term U.S. CMS
commitments. In particular, U.S. CMS has traditionally been a leader in 
the areas of data management and centralized production workflow management.

To meet two of the three goals outlined in the introduction, increased
efficiencies across Tier-1 and Tier-2 and integration of additional
resources outside those tiers, requires CIS to become substantially more
agile.  During LS1, substantial improvements towards a more agile data
management infrastructure were made by creating a global XRootD data
federation, and development and deployment of a dynamic data placement and
management system (DDM).  Both efforts were led by NSF-funded
personnel. Data management and workflow management development is done at
Cornell, XRootD development at UCSD, and DDM development at MIT.
Additional support was provided by NSF through the ``Any Data, Anytime,
Anywhere'' (AAA) project to Nebraska, UCSD and Wisconsin.

%In the following, we briefly summarize those recent achievements, and then
%describe necessary future work to be done within the present proposal.

\subsubsection{Current Status}

\noindent{\bf Dynamic Data Placement and Management (DDM):}
The PhEDEx data distribution system, which underpins the CMS computing
infrastructure, was largely manually operated during Run~1.    Operators
moved large chunks of data on command across the grid as necessary to
enable subsequent processing.  This labor intensive and lead to inefficient use
of disk resources.
The DDM software was developed and commissioned during LS1 to address these
shortcomings. DDM is now deployed at all tiers to automatically place data
where it is meant to be processed at the time it needs to be there, and
then prune the unused but archived data using well-defined policies.  
%
%For example, the archived full event, {\it i.e.} raw plus reconstructed
%quantities (FEVT), is pruned from disk to keep sufficient space for the
%Tier-0 and Tier-1 reconstruction workflows to execute smoothly. Most
%importantly, we are able to keep at least one copy of all AOD on disk
%somewhere in the CMS data federation, and duplicate multiple times as
%needed for popular data.
DDM uses PhEDEx to manage the actual transfers. It thus replaces decisions
made by human operators during Run~1 with algorithms based on dataset
``popularity'', {\it i.e.} use.

\noindent{\bf CMS Data Federation (AAA):}
CMS Data Federation is built to provide seamless international-scale data
access under the auspices of the AAA project. AAA removes the requirement
of co-location of storage and processing resources through an
infrastructure that is transparent to users and highly reliable.  It
enables greater access to the data, 
%in that users no longer have the burden
%of purchasing and operating complex disk systems to serve their
%processors. 
in fact, {\bf A}ny data can be accessed {\bf A}nytime from {\bf A}nywhere (AAA) with an
internet connection. 
%The key to success of AAA is the improved wide-area
%network access due to enhancements made to our dedicated LHC network.
AAA is made possible by XRootD software, which allows the creation of
data federations. A data federation serves a global namespace via a
tree of XRootD servers. 
The CMS data federation is now fully deployed across all tiers of the global
computing infrastructure. 

%Easy access to this data federation across the
%wide-area network is democratizing the computing abilities of university
%groups across the world because sites need not store any data
%locally. Local campus clusters controlled by non-CMS entities are easily
%integrated in the CMS computing environment. Temporary access to dedicated
%large resources can be purchased on commercial clouds or obtained from
%national or campus research facilities.

%The main advantages of AAA over DDM are that AAA fully supports partial
%file reads (a typical analysis job accesses less than 10\% of the data in
%each file) and ease of integration of non-traditional resources. The
%drawback of AAA as deployed today is that data still needs to be placed
%somewhere in the world first, and replication of datasets need to be
%controlled according to needs.  The two systems thus complement
%each other perfectly.

\noindent{\bf Improvements to CMS Workflows:}
The main objectives of the workflow management middleware is to
process data as quickly as possible, maintain uniform load across all
resources and enable fast recovery in case of service
interruptions.
%, e.g., by relocating jobs on an alternate site, while
%keeping track of the integrity of the combined dataset.  
During Run 1 each tier was used for only a small subset of all workflows.
%This led to inefficiencies and delays in processing due to inflexibility.
During LS1, we expanded dramatically what workflows can be run where,
making the overall system much more flexible, thus decreasing processing delays due to inefficiencies. 
In addition, all resource
usage, distributed analysis and centralized production, is now scheduled
via a single global HTCondor pool, allowing for relative prioritization of
different activities.
%\noindent{\bf Using Resources Beyond the Tiered System}
At this point, CMS is in an exploratory phase for smoothly integrating resources beyond the tiered system
for production and routine use. Such resources include the CMS High Level Trigger Farm (HLT) whenever the DAQ is not running, 
DOE and NSF HPC facilities like NERSC and SDSC, commercial cloud resources such as AWS, and campus IT resources
across the nation.

\subsubsection{Future Plans}

The PhEDEx data management system has served CMS extremely well for more
than ten years.  However, it is ill-equipped for the more agile needs of
the future. Its internal mechanisms for source selection for a given
transfer is much less agile than the bit-torrent-like multi-source client
used in XRootD. Its internal back-off mechanism for handling transfer
failures leads to long transfer tails, and it is generally very difficult
if not impossible to fill modern large-bandwidth network pipes using
PhEDEx.  A re-design of PhEDEx to address these issues also provides an
opportunity to consider discontinuing poorly supported protocols like SRM,
as well as duplicative protocols such as gridftp when XRootD is required
for agile operations.  The MIT group will take
on this task.

The AAA toolset contains a proxy-cache that is not yet used in CMS.
Initial tests indicate that the cache system performs exceptionally
well. As we gain more experience with this technology, we may want to
transition a sizable fraction of the U.S. CMS disk space at Tier-2s and
Tier-3s into XRootD proxy caches to gain additional efficiencies.
Development work to this end will be pursued at UCSD.

The entire end-to-end centralized data production process still has far too
many human effort intensive aspects.  Significantly more automation is
needed to make the overall system both more agile and more efficient. 
%Today it is still not uncommon that some resources, especially in U.S. CMS, are
%oversubscribed while others, especially elsewhere in the world, remain
%unused.  Enforcing dynamically changing processing priorities is also still
%very difficult due to multiple layers of queuing and workflow
%restrictions. 
%Significant efficiencies are yet to be gained here.  
Efforts on workflow management systems will continue at Cornell and Purdue.

Finally, commissioning of resources beyond the tiered system is still at
the very beginning, and while having large potential for additional
resources, will require significant effort still.  Wisconsin will
work on these issues.

\subsection{Software and Support (WBS 5)}

Multicore computing systems have become ubiquitous in the past
decade. However, efficient use of available resources, especially memory
volume and access, require adaptation of our software to suitable
multithreaded frameworks. Keeping up with technology evolution in the
market requires continuous investigation and CMS framework and utilities
software development.  The Cornell, Princeton, Florida and UCSD groups are engaged
with central CMS in this essential software development and support.
Recent efforts in software at institutions supported by the current NSF award
have focused on a number of well contained projects.  The current status of those and future plans regarding them are described in the subsections below.  

\subsubsection{Current Status}
\noindent{\bf Development of Multi-threaded CMSSW Applications:}
A systematic effort to make the core CMSSW, and the reconstruction application thread-safe has been successfully completed and deployed in the past year.  Both the HLT and Tier0 were able to use this work to great advantage. In addition to the development of the framework itself, deployment of applications requires making the algorithmic physics code "thread friendly", meaning adjusting it to follow the rules of the multi-threaded framework, such that the entire application can safely run with multiple concurrent threads.  This past year developers at Cornell have been involved in this work.
\noindent{\bf Support and Maintenance of Fireworks:}
The event display for CMS has been reworked to adapt to the DAQ system of Run 2 and to work on a variety of platforms conveniently.
\noindent{\bf Support and Maintenance Build Systems:}
In order to take advantage of developments on alternative architectures as outlined in the R\&D section there must be support in the build tools for cross compilation and/or ports to these new architectures. This past year Princeton personnel have had major success in porting both CMSSW and the OSG stack to ARM64 and Power8.
\noindent{\bf Development of MiniAOD:}
The development of the MiniAOD format described in Section~\ref{computingmodeldetails}.  
The MiniAOD is now the main data format that has been used for early results on the 13TeV data from CMS.

\subsubsection{Future Plans}
\noindent{\bf Development of Multi-threaded CMSSW Applications:}
Future steps in this area are the extension of the number of other CMSSW applications that can be run multi-threaded.  In 2016 we are still on target to make the digitization thread friendly.  This work is being carried out at Cornell.  As CMSSW evolves, constant attention must be given to keep the code thread friendly.  Personnel at Florida will be helping with this task.
\noindent{\bf Support and Maintenance of Fireworks:}
As Fireworks is the main event display for CMS, new feature requests, and updates for new operating systems are needed.  Personnel at UCSD will continue to provide this support. 
\noindent{\bf Support and Maintenance Build Systems:}
As innovations in new architectures and techniques from the R\&D area become beneficial to the operating program, it is important to have personnel that can make that transition happen.  Princeton personnel have demonstrated success in this area, and will continue to provide this support.
%LSK: it appears that this work is moving to WBS 6 which is fine.  I would claim that maintenance of this development is now in the hands of our Italian colleagues, so I comment it out of this program of work: 
%\noindent{\bf Development of MiniAOD}
%The development of the MiniAOD format described in Section~\ref{computingmodeldetails}.  
%The MiniAOD is now the main data format that has been used for early results on the 13TeV data from CMS.

\subsection{Technologies and Upgrade R\&D (WBS 6)}

The R\&D effort within the project aims to control the rate of growth of 
computing required, and thus cost incurred, and to retain flexibility for 
future changes in computer architectures and software technologies. This 
relatively small WBS area aims to leverage for CMS developments from 
elsewhere in HEP and beyond. 

Two largely independent effects drive cost:
those that scale with event complexity, or average pileup (PU) per event, 
and those that scale with integrated luminosity, or total data volume.
The cost of event reconstruction is driven by occupancy of the tracker.
Higher instantaneous luminosity leads to higher pileup and occupancy, and
with it a near exponential growth in CPU time per event in the pattern
recognition step of the track reconstruction.  For example, an increase in
the average number of PU events from 20 to 30 was measured to result in a
three-fold increase of event reconstruction time in the current CMS software
release.  This range of PU matches the expected running conditions during
Run 2.  To set the scale, the time to reconstruct a 13~TeV $t\bar{t}$ event
exceeds the time to simulate the same event at an average PU $\sim$ 25,
which we expect to reach in 2016/17.  Any speed-ups of the
reconstruction software, especially the tracking pattern recognition, thus
directly translate into computing cost savings.

Analysis, MC production, and data reprocessing all scale roughly linearly
with total integrated luminosity, or total data volume, leading to the
$\times 30$ increase from the beginning of Run~2 in 2015 to the end of
Run~3 in 2023 mentioned previously.  The software and computing R\&D program
for the next five years is geared towards two timelines. It is meant to engage
in fundamental R\&D towards solving the challenges of scaling out computing
for the HL-LHC era (2025-2035) but also to provide near term
improvements that can be put into production during LS2 (2019-20) in order
to address the challenges of Run 3 (2021-23). Given limited resources in
personnel, our strategy is to focus on the long term with an eye towards
adopting lessons learned in this process to address the Run~3 challenges.

\subsubsection{Current Status}

During LS1, US-CMS drove multiple developments all focused on overall cost reductions in computing.
We led the algorithmic improvements and code optimizations in the pattern recognition software that reduced the reconstruction
time per event by xY for an average PU $\sim$ 30. We instigated the introduction of MiniAOD, reducing the event size by x10
and the average analysis processing time per event by x2-5.
We prepared the core framework to be multithreaded blablabla --- Sridhara to fill in text from David Lange here ... 
And we transitioned the computing infrastructure and services that CMS depends on towards a suite of services that are much 
more agile as described in Section~\ref{cis}.

\subsubsection{Future Plans}

\noindent{\bf Reconstruction Software:} Cornell, Princeton, and UCSD are
collaborating on an ambitious R\&D program to redesign the core
Kalman filter tracking algorithms of CMS for parallel architectures.  While
the bulk of the long term R\&D is funded via an
independent NSF collaborative research award (PHY-1521042,1520942,1520969), the present proposal includes 
effort to derive short-term benefits from the
independently-funded long term focused R\&D agenda.  This is particularly
interesting in light of the planned roll-out of large Supercomputers at
both DOE and NSF based on the next two generations of Intel MIC
processors. For example, Cori Phase 2 at NERSC plans for 9,300 Intel
Knights Landing processors by 2017. Aurora at ANL is expected to deploy
50,000 Intel Knights Hill processors in 2018.  Similar plans exists in the
NSF for the Stampede Supercomputer at TACC.  Whether CMS can
benefit from these large scale resources for its core processing needs in
Run~2 and Run~3 depends crucially on successfully transitioning lessons
learned from the long term R\&D program into
production. This transition is within the scope of the present proposal.

\noindent{\bf R\&D towards a new data analysis model:} For the HL-LHC era, CMS must
contemplate a fundamental shift in the boundary between ``primary data''
and ``custom data''.  Already in Run~1, the custom data Ntuples typically
were analyzed at event rates ranging from 100 Hz to 10 kHz.  Ntuple
analysis is even today in many cases I/O rather than CPU limited. In
contrast, the production of these custom Ntuples is almost always CPU
limited. Even for the MiniAOD of Run 2, typical event processing rates
reach little more than a few Hz.  There is a trade-off between flexibility
and speed. A data format for the entire CMS collaboration must be
flexible in content for two reasons. First it needs to satisfy many types
of data analyses, and second it must be ``forward compatible,'' {\it i.e.} a
MiniAOD produced today must still be useful a few months from now when the
state of the art in physics object definition have changed to incorporate improvements. 
The R\&D questions here include: Can we speed up MiniAOD to the kinds of event
processing rates typical for custom Ntuples? If we can, what does it mean
for the Tier-2 infrastructure to support I/O limited jobs at large scale?
Do we need to schedule disks for I/O limited jobs?
Can we reuse some of the industry standard ``Big Data'' products, or is this impossible because we
would loose the benefits of partial file reads in ROOT I/O?
CMS will also benefit from the separately funded NSF SI2 DIANA/HEP (Data 
Intensive ANAlysis for High Energy Physics) project (ACI-1450310, 1450319,
1450323, 1450377) involving PIs working on CMS, Atlas and LHCb.
DIANA/HEP will develop state-of-the-art software tools for analysis and improve
interoperability of HEP tools with the larger scientific software ecosystem.
A modest effort has been included in this proposal to integrate those
advances into CMS and its analysis model.

\subsection{Coordination with CMS (WBS 7)}

U.S. CMS S\&C personnel are well integrated in the CMS-wide coordination
efforts and hold management positions.  

\subsubsection{Current Status}

Current support under this category includes S\&C coordination at
Princeton, reconstruction software coordination at Wisconsin and UCSD, U.S. CMS Tier-2 program coordination
at UCSD, and submission infrastructure coordination at UCSD.

\subsubsection{Future Plans}

It is anticipated that approximately a third of the management positions
in CMS are held by U.S. personnel, of which NSF computing supported 
personnel needs will have to be covered by this project.  The need
is likely to remain approximately constant.

\end{document}

\end
